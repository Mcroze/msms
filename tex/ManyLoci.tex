\documentclass{article}
\usepackage{amssymb}
\usepackage{amsfonts}
\usepackage{amsmath}
\usepackage{amsthm}
\usepackage{color}
%\usepackage{url}
%\usepackage{graphicx}

\usepackage[perpage,symbol*]{footmisc}

%\DefineFNsymbols*{daggers}{\dagger\ddagger{\dagger\dagger}{\ddagger\ddagger}*{**}}
%\setfnsymbol{daggers}


\begin{document}
\title{Multiple Selected Loci}
\maketitle

\section{Introduction} 
This document is meant   for internal purposes. Its much easier to
communicate math and details of algorithms in a properly formatted \LaTeX
document. 

We are now considering adding multiple selected loci sooner rather than later
into msms and as such we want to ensure than we are all can clearly see the
math behind the proposed changes\footnote{And to ensure that it correct}. I do
not believe this is as a big a change as I originally though as we will see. 

I present this as a $k$ allele per loci with $n$ selected loci rather than just
two. This is because it is much easier to implement a generic model on the
computer than a particular case. In other words I am trying to follow the
imagined implementation as much as possible. 

The main assumption that is used is that the probability than a single strand of
DNA has more than a single recombination event in a single generation is low
enough to be ignored. I also do not consider gene conversion. 

\section{Forward Simulation of Selection}

\subsection{Overview and Definitions}

\newcommand{\hap}{\mathbf{h}}
\newcommand{\happ}{\mathbf{h'}}

The extension to many selected loci is done by considering haplotype
frequencies at each generation in each deme. So a forward simulation results in a
list of haplotype frequencies that the coalescent simulation pass must then be
conditional on. The coalescent extension is strait forward in that it is almost
identical to a $k$ allele model with a minor change to the recombination events.
A key difference however is that when we carry out the forward simulation we must
take recombination into account as this can alter the haplotype frequencies.
Fortunately this does not appear difficult to do\footnote{Note: difficult does
not equal fast to implement!}.

We assume that the $n$ loci are on a single strand of DNA and in natural order.
That is, locus 1 is the ``left most'' locus while locus $n$ is the right most. In
order to track $n$ loci each with $k$ alleles we consider haplotype frequencies
rather than allele frequencies at each locus. We work with selection coefficients
$w_{h,h'}$ for the $h$'th  haplotype where  $h$ is the haplotype number defined
below, and the $h'$ is the haplotype of the second strand of DNA in the diploid
case. Let the vector $\hap=(a_1,a_2,\ldots,a_n)$ denote a haplotype where $a_j$
is the $a$'th allele at the $j$'th loci. The haplotype number for
$\hap=(a_1,a_2,\ldots)$ is given by $h=\sum_i^n a_ik^{i-1}$.Let $x^d_h$ be the
frequency of the $h$ haplotype in the current generation in the $d$'th deme. We
also denote this as $x^d_{a_1,a_2,\ldots,a_n}$ for convenience, and we omit the
deme superscript where context makes it clear that we dealing with per deme
frequencies. We define the $c$ marginal frequency as:
\begin{align}
	x_{a_1,a_2,\ldots,a_c|c}&=\sum_{a_{c+1}}^k\sum_{a_{c+2}}^k \cdots
	\sum_{a_n}^k x_{a_1,a_2,\ldots,a_n} \\
	&=\sum_{a_c}^k x_{a_1,a_2,\ldots,a_{c-1}|c-1}
\end{align}
and {\color{red} WRONG! EPIC FAIL}
\begin{align}
	x_{c|a_{c+1},a_{c+2},\ldots,a_n}&=\sum_{a_1}^k\sum_{a_2}^k \cdots
	\sum_{a_c}^k x_{a_1,a_2,\ldots,a_n} \\
	&=1-x_{a_1,\ldots,a_c|c} 
\end{align}
We use the short hand $x_{\hap|c}=x_{a_1,\ldots,a_c|c}$ and
$x_{c|\hap}=x_{c|a_{c+1},\ldots,a_n}$. The proportion of recombination events
between locus $j$ and $j+1$ is denoted $R_j$ and we require
that $\sum_j R_j<1$\footnote{The assumption is in fact that $\sum_j R_j\ll 1$}.
Migration from deme $j$ to deme $i$ is defined as the proportion $m_{ij}$ of
deme $i$ that is made  up of migrants from deme $j$ and  we set
$m_{ii}=1-\sum_{j,j \neq i} m_{ij}$. 

We avoid adding time or deme dependence on selection coefficients as this is quite
a trivial extension while it would clutter the notation. Similar extensions
with varying population sizes and deme structure also exist and are equally
somewhat trivial extensions.

\subsection{Life cycle}

I assume the exact life cycle that is used will make minimal difference to the
simulations compared to other approximations made. However this may not be the
case, and hence I try to be very clear with how I see the current life cycle.
Note that I will consider both haploid and diploid populations since both are
important in theory and practice. 

The life cycle is as follows (assuming diploid for now).
\begin{enumerate}
  \item  We assume that we have HWE and we now work with an
  infinite population size. 
  \item Recombination. This causes a change in haplotype frequencies. But not
  allele frequencies. We now consider the resultant haplotype frequencies to be
  the gamete frequencies.
  \item Mutation. Both haplotype frequencies and allele frequencies can change.
  \item Random mating and Selection. With the assumption of HWE this changes
  haplotype frequencies and allele frequencies. We still use the infinite
  population size approximation. 
  \item Migration. We now reconstruct the deme haplotype frequencies taking
  migration into account. 
  \item Sampling. Finally we move from the infinite population size to a finite
  population size with multinomial sampling, there by incorporating drift. 
\end{enumerate}
By assuming HWE we can track haplotype frequencies rather than genotypes for
diploid cases.

In practice we follow these steps quite closely in code\footnote{Or will
when I write the code}. This makes the code easy to follow, does not effect
performance, and makes it much easier to extend and expand.

\subsection{Recombination} 

Consider the frequencies $x'_{a_1,\ldots,a_n}$ of a given haplotype after
recombination and we set $\hap=(a_1,\ldots,a_n)$. First lets consider
recombinant between the first and second locus. The proportion of the population that has this recombination event is
$R_1$ where $x_{a_1|1}$ have the correct allele at the first locus and
$x_{1|a_2,\ldots,a_n}$ have the correct allele's at the remaining loci. So the
proportion of newly formed $\hap$ haplotypes from recombination at between the
first locus and second locus is:
\begin{align}
	x'_{R_1,\hap}&=R_1 x_{a_1|1} x_{1|a_2,\ldots,a_n} \\
	&=R_1 x_{a_1|1} (1-x_{a_1|1}) \\
	&=R_1 x_{\hap|1} (1-x_{\hap|1})
\end{align}
Summing over all recombination events and considering the proportion of $h$
haplotypes that do not recombine this generation, we get:
\begin{align}
	x'_{\hap}=R x_{\hap}+\sum_{j=1}^{n-1} R_j x_{\hap|j} (1-x_{\hap|j})
\end{align}
Where $R=1-\sum_j R_j$ is the proportion of individuals that do not recombine
this generation. 

Recall that the assumption here is that recombination does not  ``cut'' the DNA
in two places in any one generation. Otherwise we would have to consider
$x_{c|\hap|c'}$ type marginal frequencies. However these events would happen
with probability $R_cR_{c'}$ and for many cases can be ignored. Also I am ignoring
gene conversion which would also require considering $x_{c|\hap|c'}$ marginal
frequencies\footnote{This can be done of course}.

\subsection{mutation}

Mutation is easy to implement while just considering all haplotype to haplotype
transitions with a hamming weight of 1. That is at most 1 locus changes between
haplotypes. Multiple mutations are ignored which is a good approximation with
realistic mutation rates. We don't bother with a formula. 


\subsection{Selection}

Selection is just a standard $k$ alleles selection step only the total number
of ``alleles'' in this case is $nk$. We use the haplotype number $h$ rather
than alleles for this derivation. To simplify notation\footnote{perhaps abuse
of notation is a better word} we let $x$ now denote the  frequencies after
recombination, and $x'$ denote the frequencies after selection. 

\subsubsection{haploid}

This case is much simpler than the diploid case and as such we complete this
first. As defined above we let the selection coefficient for the $h$ haplotype
be $w_h$. Restricting our discussion to a single deme, the amount of haplotype
$h$ after selection is simply $x_hw_h$. Including normalization we get:
\begin{align}
	x'_h=\frac{x_hw_h}{\sum_z x_hw_h}
\end{align} 

\subsubsection{diploid}

We now must consider homozygotes as well as heterozygous. We let selection
coefficients be defined for the full genotype, so $w_{h,h'}$ is the selection
coefficient for the genotype $h,h'$ and we require that $w_{h,h'}=w_{h',h}$. The
frequency $x_h$ of the $h$ haplotype after selection but before migration is:
\begin{align}
	x'_h=\frac{x_h \sum_z w_{h,z}x_z}{\sum_{z'} x_{z'}\sum_z w_{z',z}x_z}
\end{align}

\subsection{Migration}

Again we consider that $x_h$ now denotes the frequency after both recombination
and selection but before migration of the $h$ haplotype. While $x'_h$ will
denote the frequency after migration. Directly from the definition of $m_{ij}$
we can write down the frequencies after migration as:
\begin{align}
	x'^d_h=\frac{\sum_j m_{dj} x^j_h}{\sum_z\sum_j m_{dj} x^j_z}
\end{align}

\section{Coalescent}

Coalescent simulation are conditional haplotype frequencies. Migration and
coalescent remain unchanged from the single selected bi-allelic locus. Recurrent
mutation as an event is also easy to account for, however one must consider
stopping conditions as we no longer have a clear case of single direction
mutation (some loci/allele may be deleterious). However this is a
implementation detail that is less critical since single ``fixation'' is a more
special case in a many allele many loci model anyway. 

The remaining event that does need some modification is recombination. However
again this is not difficult. We consider a single lineage as we move past-ward
in time, and is a member of the $\hap$ haplotype that has a recombination event.
A cut point is generated at random according to some recombination model. It
cuts the strand between the $l$'th and $l+1$ selected loci. The left part of
the strand with the first $l$ loci alleles already fixed randomly picks one of
the haplotypes that has the same alleles at these loci with probability
proportional to the frequencies. That is the probability that the left strand will
now be in the $\happ$ haplotype is (conditional that $\happ$ and $\hap$ have the
same alleles for first $l$ loci)\footnote{I am running out of notation ideas}:
\begin{align}
	\Pr\{\hap \to \happ\}=\frac{x_{\happ}}{x_{\hap|l}}
\end{align}
A similar result holds for the right side of the cut(conditional that $\happ$
and $\hap$ have the same alleles for last $n-l$ loci):
\begin{align}
	\Pr\{\hap \to \happ\}=\frac{x_{\happ}}{x_{l|\hap}}
\end{align}

\end{document}
